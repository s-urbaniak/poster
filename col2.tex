\begin{block}{Methode}
\begin{columns}[t,totalwidth=\twocolwid]

% {{{ column 2.1
\begin{column}{\onecolwid}\vspace{-.6in}
\centering Studie 1
\linebreak
\begin{itemize}
\justifying
\item N=208
\item Sechs dritte Klassen 3 x 6h Unterrichteinheiten in \textbf{Mathematik}.
\item \textbf{Arbeitsmaterialien}: Arbeitshefte mit eigenständig zu bearbeitenden Aufgaben, Anschauungsmaterialien zum Betrachten, Basteln und Experimentieren
\item Vergleich des Lernerfolges  mit \textbf{drei} lehrergeleitet unterrichteten \textbf{Kontrollklassen}
\end{itemize}
\end{column}
% }}}

% {{{ column 2.2
\begin{column}{\onecolwid}\vspace{-.6in}
\centering Studie 2
\linebreak
\begin{itemize}
\justifying
\item N=54
\item Eine dritte Klasse  im Verlauf  eines Schuljahres 6 x 6 h Unterrichteinheiten im \textbf{Sachunterricht}.
\item \textbf{Arbeitsmaterialien}: Reader mit Texten und Abbildungen, ein Aufgabenheft mit Leitfragen für beide Phasen, ein kleines, Experiment für jeden Expertenbereich.
\item Vergleich des Lernerfolgs mit \textbf{einer} lehrergeleitet unterrichteten \textbf{Kontrollklasse}
\end{itemize}
\end{column}
% }}}

\end{columns}
\end{block}

% {{{ Auswertungsverfahren
\begin{block}{Auswertungsverfahren}
\begin{enumerate}
\item Indikatoren für den Wissenszuwachs - Individuelle Differenzwerte zwischen Vor- und Nachkenntnisleistungen.
\item Überprüfung des Experteneffektes - Bestimmung von mittleren Wissenszuwächsen der Experten, der Nichtexperten und lehrergeleiteten Lernenden.
\item Vergleichen der Testwerte über die Effektgrößen (d)
\end{enumerate}
\end{block} 
% }}}

\begin{block}{Ergebnisse} % This block contains columns 2.3, 2.4
\begin{columns}[t,totalwidth=\twocolwid] % Split up the two columns wide column again

% {{{ column 2.3
\begin{column}{\onecolwid}
\begin{table}
\begin{tabular}{lccc}
 & \multicolumn{3}{c}{Studie 1} \\
\cmidrule{2-4}
 & d\textsubscript{Experten - Nichtexperten} & d\textsubscript{Experten - Kontrolle} & d\textsubscript{Nichtexperten - Kontrolle} \\
\midrule 
\textbf{Ma 1} & 0.24 & 0.37 & 0.04 \\
\textbf{Ma 2} & 0.34 & -0.02 & -0.32 \\
\textbf{Ma 3} & 0.26 & 0.41 & -0.07 \\
\midrule 
\textbf{Gesamt} & 0.28 & 0.25 & -0.12 \\
\bottomrule
\end{tabular}
{\caption*{Anm.: Positive Werte bedeuten jeweils eine Überlegenheit der vorne stehenden Gruppierung.}}
\end{table}

\justifying
Frage 1: Die Experten sind  den Nichtexperten gegenüber, sowie in der ersten und dritten Unterrichtseinheit, gegenüber den nicht kooperativ unterrichteten Kontrollklassen überlegen.
 
Frage 2: Die Nichtexperten kamen in ihren Lemzuwächsen über das Niveau 
der Kontrollklassenschüler nicht hinaus.

Am meisten gelernt haben die Experten, dann die Kinder aus den Kontrollklassen und schließlich die Nichtexperten
 
\end{column}
% }}}

% {{{ column 2.4 
\begin{column}{\onecolwid}
\begin{table}
\begin{tabular}{lccc}
 & \multicolumn{3}{c}{Studie 2} \\
\cmidrule{2-4}
 & d\textsubscript{Experten - Nichtexperten} & d\textsubscript{Experten - Kontrolle} & d\textsubscript{Nichtexperten - Kontrolle} \\
\midrule 
\textbf{Sa l} & 0.68 & 1.21 & 0.52 \\
\textbf{Sa 2} & 0.41 & 0.23 & -0.09 \\
\textbf{Sa 3} & 0.40 & 0.79 & 0.28 \\
\textbf{Sa 4} & 1.16 & 0.29 & -0.89 \\
\textbf{Sa 5} & 0.76 & 0.35 & -0.48 \\
\textbf{Sa 6} & 0.74 & 0.72 & -0.05 \\
\midrule 
\textbf{Gesamt} & 0.69 & 0.60 & -0.12 \\
\bottomrule
\end{tabular}
{\caption*{Anm.: Positive Werte bedeuten jeweils eine Überlegenheit der vorne stehenden Gruppierung.}}
\end{table}

\justifying
Frage 1: Die Experten zeigen in ihren Themen eine deutliche Überlegenheit gegenüber den Nichtexperten und der Kontrollgruppe.

Frage 2: Der Lernzuwachs der Nichtexperten unterscheidet sich nicht bedeutsam von dem der Kontrollklassenschüler.

Frage 3: Keine Verbesserung der Vermittlungskompetenzen über die Zeit.
\end{column}
% }}}

\end{columns}
\end{block}

\begin{block}{Diskussion}
Die Experteneffekt wurde in beiden Studien bestätigt. Lernerfolge sind vor allem auf die Vermittlungsphase zurückzuführen. Im Primarbereich gibt es Probleme in den Vermittlungsphasen. ( bis 9 Jährigen brauchen intensivere Förderung um das kompetente „Erklären-Können“ zu verbessern. Vorschläge: Evaluation des Gruppenarbeitsprozesses („group prozes sheat“: Aronson, Patnoe, 1997). Direkte Förderung der Vermittlungskompetenzen der Kinder und deren Interaktion untereinander ("Guided Peer Questioning": King, 1999)
\end{block}
