\begin{columns}[t,totalwidth=\twocolwid] % Split up the two columns wide column
\begin{column}{\onecolwid}\vspace{-.6in} % The first column within column 2 (column 2.1) {{{
\begin{block}{Studie 1}
\begin{itemize}
\justifying
\item Sechs dritte Klassen 3 x 6h erhielten Unterrichteinheiten in \textbf{Mathematik} nach der Methode des Gruppenpuzzles.
\item \textbf{Arbeitsmaterialien}: Arbeitshefte mit eigenständig zu bearbeitenden Aufgaben, Anschauungsmaterialien zum Betrachten, Basteln und Experimentieren.
\item Lernerfolg würde mit \textbf{drei} lehrergeleitet unterrichteten \textbf{Kontrollklassen} vergleichen.
\end{itemize}
\end{block}
\end{column} % }}}
\begin{column}{\onecolwid}\vspace{-.6in} % The second column within column 2 (column 2.2) {{{
\begin{block}{Studie 2}
\begin{itemize}
\justifying
\item Eine dritte Klasse erhielt im Verlauf des Schuljahres 6 x 6h Unterrichteinheiten im \textbf{Sachunterricht} nach Methode des Gruppenpuzzles. 
\item \textbf{Arbeitsmaterialien}: Reader mit Texten und Abbildungen, ein Aufgabenheft mit Leitfragen für beide Phasen, ein kleines, von den Kindern durchzuführendes  Experiment für jeden Expertenbereich.
\item Lernerfolg würde mit \textbf{einer} lehrergeleitet unterrichteten \textbf{Kontrollklasse} vergleichen
\end{itemize}
\end{block}
\end{column} % }}}
\end{columns} % End of the split of column 2 - any content after this will now take up 2 columns width

\begin{block}{Auswertungsverfahren}
Als Indikatoren für den Wissenszuwachs werden die individuellen Differenzwerte zwischen Vor- und Nachkenntnisleistungen herangezogen. Um Experteneffekt überprüfen, werden für jeden Expertenbereich einer Unterrichtseinheit die mittleren Wissenszuwächse der Experten, der Nichtexperten und der lehrergeleitet Lernenden bestimmt. Da die Testwerte der Schülerinnen und Schüler in den verschiedenen Expertenbereiche nicht direkt miteinander vergleichbar sind, werden die jeweiligen Effekte über die Effektgrößen (d) bestimmt.
\end{block} 

\begin{block}{Ergebnisse} % This block will contain the columns 2.3, 2.4
\begin{columns}[t,totalwidth=\twocolwid] % Split up the two columns wide column again
\begin{column}{\onecolwid} % The third column within column 2 (column 2.3) {{{
\begin{table}
\begin{tabular}{lccc}
 & \multicolumn{3}{c}{Studie 1} \\
\cmidrule{2-4}
 & d\textsubscript{Experten - Nichtexperten} & d\textsubscript{Experten - Kontrolle} & d\textsubscript{Nichtexperten - Kontrolle} \\
\midrule 
\textbf{Ma 1} & 0.24 & 0.37 & 0.04 \\
\textbf{Ma 2} & 0.34 & -0.02 & -0.32 \\
\textbf{Ma 3} & 0.26 & 0.41 & -0.07 \\
\midrule 
\textbf{Gesamt} & 0.28 & 0.25 & -0.12 \\
\bottomrule
\end{tabular}
{\caption*{Anm.: Positive Werte bedeuten jeweils eine Überlegenheit der vorne stehenden Gruppierung.}}
\end{table}

\justifying
Die Effektgrößen zeigen eine moderate, durchgängige pädagogisch bedeutsame Überlegenheit der Experten gegenüber den Nichtexperten sowie in der ersten und dritten Unterrichtseinheit gegenüber den nicht kooperativ unterrichteten Kontrollklassen (Fragestellung 1). Die Nichtexperten kamen in ihren Lemzuwächsen über das Niveau der Kontrollklassenschüler nicht hinaus. (Fragestellung 2). 

Am meisten gelernt haben die Experten, dann die Kinder aus den Kontrollklassen und schließlich die Nichtexperte.

\end{column} % }}}
\begin{column}{\onecolwid} % The fourth column within column 2 (column 2.4) {{{
\begin{table}
\begin{tabular}{lccc}
 & \multicolumn{3}{c}{Studie 2} \\
\cmidrule{2-4}
 & d\textsubscript{Experten - Nichtexperten} & d\textsubscript{Experten - Kontrolle} & d\textsubscript{Nichtexperten - Kontrolle} \\
\midrule 
\textbf{Sa l} & 0.68 & 1.21 & 0.52 \\
\textbf{Sa 2} & 0.41 & 0.23 & -0.09 \\
\textbf{Sa 3} & 0.40 & 0.79 & 0.28 \\
\textbf{Sa 4} & 1.16 & 0.29 & -0.89 \\
\textbf{Sa 5} & 0.76 & 0.35 & -0.48 \\
\textbf{Sa 6} & 0.74 & 0.72 & -0.05 \\
\midrule 
\textbf{Gesamt} & 0.69 & 0.60 & -0.12 \\
\bottomrule
\end{tabular}
{\caption*{Anm.: Positive Werte bedeuten jeweils eine Überlegenheit der vorne stehenden Gruppierung.}}
\end{table}

\justifying
Die Experten zeigen in ihren Themen eine deutliche Überlegenheit gegenüber den Nichtexperten und den Schülern der Kontrollgruppe (Fragestellung 1). 
Der Lernzuwachs der Nichtexperten unterscheidet sich nicht bedeutsam von dem der Kontrollklassenschüler (Fragestellung 2).
In zwei der sechs Unterrichtseinheiten sind die Nichtexperten den Kindern der Kontrollklasse überlegen. 

\end{column} % }}}
\end{columns} % End of the split of column 2
\end{block}

\begin{block}{Diskussion}
Hinsichtlich der Lernleistungen  belegen Metaanalysen die positiven Effekte kooperativer Lehr- und Lernmethoden (Johnson, Johnson \& Stanne, 2000; Rohrbeck, Ginsburg-Block, Fantuzzo \& Miller, 2003; Slavin 1995). Die für die kooperativen Methoden berichteten Effekte variieren allerdings stark, und über Wirkmechanismen des 
\end{block}
